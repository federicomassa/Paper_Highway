%\begin{IEEEbiography}[{\includegraphics[width=1in, height=1.25in, clip, keepaspectratio]{Images/fake-photo}}]{Adriano Fagiolini} received his master's degree in computer science engineering, awarded {\em cum laude} in 2004, from the University of Pisa, with a thesis on casting manipulation. He was summer student at the European Center for Nuclear Research (CERN), Geneva. He received his Ph.D. degree in robotics and automation from the University of Pisa in 2009. During his Ph.D., he enrolled in the International Curriculum Option of doctoral studies in hybrid control for complex, distributed and heterogeneous embedded systems, and he has actively participated in various European Union Projects. He led the University of Pisa Team at the first European Space Agency's Lunar Robotics Challenge, where the team came in second place. His research interests include Boolean consensus and intrusion detection in distributed robotic systems. He is carrying on his activity at the University of Pisa, where he serves as a postdoc researcher.
%\end{IEEEbiography}
%
%\begin{IEEEbiography}[{\includegraphics[width=1in, height=1.25in, clip, keepaspectratio]{Images/fake-photo}}]{Gianluca Dini} received the Laurea degree in Electronic Engineering from the University of Pisa in 1990 and a Ph.D. in Computer Engineering from Scuola Superiore S. Anna, Pisa, in 1995. Since 2000, he is Associate Professor of Computer Engineering at the University of Pisa. His main research interests are in distributed computing with particular reference to security and fault-tolerance.
%\end{IEEEbiography}
%
%\begin{IEEEbiography}[{\includegraphics[width=1in, height=1.25in, clip, keepaspectratio]{Images/fake-photo}}]{Antonio Bicchi} is Professor of System Theory and Robotics at the University of Pisa. He graduated at the University of Bologna in 1988 and was a postdoc scholar at M.I.T. A.I. Lab in 1988--1990. His main research interests are in Dynamics, kinematics and control of complex mechanical systems, including robots, autonomous vehicles, and automotive systems; Haptics �and dextrous manipulation; Theory and control of nonlinear systems, in particular hybrid (logic/dynamic, symbol/signal) systems. He has published more than 200 papers on international journals, books, and refereed conferences. He currently serves as the Director of the Interdepartmental Research Center ``E. Piaggio'' of the University of Pisa, and as Editor in Chief of the Conference Editorial Board for the IEEE Robotics and Automation Society (RAS). Antonio Bicchi is an IEEE Fellow since 2005. He has served as Vice President of IEEE RAS, Distinguished Lecturer, and editor for several scientific journals including Transactions on Robotics and Automation and Int.l J. Robotics Research. He has organized and co-chaired the first WorldHaptics Conference (2005) and Hybrid Systems: Computation and Control (2007).
%\end{IEEEbiography}
